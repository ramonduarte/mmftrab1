%%%%%%%%%%%%%%%%%%%%%%%%%%%%%%%%%%%%%%%%%
% Lachaise Assignment
% LaTeX Template
% Version 1.0 (26/6/2018)
%
% This template originates from:
% http://www.LaTeXTemplates.com
%
% Authors:
% Marion Lachaise & François Févotte
% Vel (vel@LaTeXTemplates.com)
%
% License:
% CC BY-NC-SA 3.0 (http://creativecommons.org/licenses/by-nc-sa/3.0/)
% 
%%%%%%%%%%%%%%%%%%%%%%%%%%%%%%%%%%%%%%%%%

%----------------------------------------------------------------------------------------
%	PACKAGES AND OTHER DOCUMENT CONFIGURATIONS
%----------------------------------------------------------------------------------------

\documentclass{article}

\input{structure.tex} % Include the file specifying the document structure and custom commands

%----------------------------------------------------------------------------------------
%	ASSIGNMENT INFORMATION
%----------------------------------------------------------------------------------------

\title{MAE001: Modelagem Matemática em Finanças I} % Title of the assignment

\author{Ramon Duarte de Melo\\ \texttt{ramonduarte@poli.ufrj.br}} % Author name and email address

\date{Universidade Federal do Rio de Janeiro (UFRJ) --- \today} % University, school and/or department name(s) and a date

%----------------------------------------------------------------------------------------

\begin{document}

\maketitle % Print the title

%----------------------------------------------------------------------------------------
%	INTRODUCTION
%----------------------------------------------------------------------------------------

\section*{Introdução} % Unnumbered section

O objetivo do Projeto I é implementar o modelo binomial de precificação computacionalmente, realizando comparações de cunho matemático-estatístico e produzindo gráficos com tais observações. 

Para tal, foi utilizada a linguagem \emph{Python 3.6.7}, com os módulos \emph{numpy} (métodos numéricos) e \emph{matplotlib.pyplot} (visualização de dados).
O programa requer a instalação destes módulos, mas possui uma ferramenta de instalação automatizada das dependências (\emph{pipenv}). 

O código utilizado neste trabalho, bem como o deste relatório e as imagens geradas, foi aberto e disponibilizado publicamente no repositório https://github.com/ramonduarte/mmftrab1.


%----------------------------------------------------------------------------------------
%	PROBLEM 1
%----------------------------------------------------------------------------------------

\section*{1ª Questão} % Numbered section

O tempo final $T = 20$ e o intervalo de discretização $\Delta t = 0.5$ foram fixados pelo enunciado da questão.
Os demais parâmetros foram escolhidos de forma a facilitar a visualização dos dados plotados.
A utilização de uma escala logarítmica teria sido uma opção mais adequada, porém os gráficos já haviam sido feitos quando esta possibilidade foi levantada pelo professor.

De toda maneira, a visualização e a observação da teoria não ficou prejudicada com a escolha destes parâmetros, que foram $S_{0} = 100$, $u = 1.024$, $d = 1/u = 0.9765625$ e $p = q = 0.5$. 
Como não havia a necessidade de calcular a depreciação dos ativos, a taxa da renda fixa foi considerada como $r = 0$.
Embora este dado não seja necessário em cenários reais (não-neutros a risco), o valor esperado de crescimento dos ativos é zero, isto é, $E[S(n)] = E[S(n+1)]$.
Portanto, o valor esperado para a média é o mesmo que o inicial do ativo: $E[\bar{S}] = E[S_{0}]$.
Foram gerados 20 passeios aleatórios baseados nestes parâmetros.

\begin{figure}[]
	\includegraphics[width=\linewidth]{Figure_1.png}
	\centering
	
	\caption{Passeios aleatórios com $S_{0} = 100$, $u = 1.024$, $d = 1/u = 0.9765625$ e $p = q = 0.5$.}
	\label{}
\end{figure}

%----------------------------------------------------------------------------------------
%	PROBLEM 2
%----------------------------------------------------------------------------------------


\section*{2ª Questão}

Os boxplots foram gerados usando o módulo \emph{matplotlib.pyplot.boxplot}.
Para facilitar a comparação dos boxplots, eles foram agrupados em subplots no mesmo gráfico.
\emph{Outliers} foram marcados com quadrados vermelhos.

\begin{figure}[H]
	\includegraphics[width=\linewidth]{fig1_250.png}
	\centering
	
	\caption{Boxplots com $250$, $500$, $1000$ e $2000$ iterações, respectivamente, de baixo para cima.}
	\label{}
\end{figure}

%----------------------------------------------------------------------------------------
%	PROBLEM 2
%----------------------------------------------------------------------------------------

\section*{3ª Questão}

Para esta questão, os boxplots foram refeitos utilizando os parâmetros $\Delta t = 0.25$ e $u = 1/d = \sqrt{1.024}$.
Esta configuração gerou uma variância consideravelmente menor, bem como uma quantidade significativamente menor de \emph{outliers}.
Embora todos os ativos pudessem atingir as mesmas taxas de crescimento e decrescimento que no cenário anterior - apesar das taxas serem menores, o maior número de iterações compensa proporcionalmente a redução -, foram raros os ativos que obtiveram mais de 30\% de variação, um cenário bastante comum na segunda questão.

Este fenômeno foi observado porque, como $E[\bar{S}] = E[S_{0}]$ e $E[S(n)] = E[S(n+1)]$, um número maior de repetições tende a aproximar os resultados da média.
A redução da variância em distribuições probabilísticas por conta da ampliação do espaço amostral é bastante conhecida e documentada. 

\begin{figure}[H]
	\includegraphics[width=\linewidth]{fig3_250.png}
	\centering
	
	\caption{Boxplots com $250$, $500$, $1000$ e $2000$ iterações, respectivamente, de baixo para cima.}
	\label{}
\end{figure}


%----------------------------------------------------------------------------------------
%	PROBLEM 2
%----------------------------------------------------------------------------------------

\section*{4ª Questão}

O valor esperado de crescimento dos ativos com estes parâmetros é zero, isto é, $E[S(n)] = E[S(n+1)]$.
Portanto, o valor esperado para a média é o mesmo que o inicial do ativo: $E[\bar{S}] = E[S_{0}]$.

Os resultados obtidos convergem fortemente para o previsto na teoria.
A maior variação da média dos passeios aleatórios foi inferior a 3\%, no boxplot $n = 250$ da terceira questão.

%----------------------------------------------------------------------------------------


%----------------------------------------------------------------------------------------
%	PROBLEM 2
%----------------------------------------------------------------------------------------

\section*{5ª Questão}

Para esta questão, algumas adaptações fizeram-se necessárias, haja vista os parâmetros selecionados inicialmente.
Em primeiro lugar, o uso da escala logarítmica aqui dificultaria a visualização dos dados, já que a média, as probabilidades e as taxas foram selecionadas de forma a facilitar a visualização em escala linear.
Em segundo lugar, foi calculado o erro quadrático médio em vez do módulo do erro médio.
Na prática, isto não faz nenhuma diferença, exceto que o gráfico observado será uma parábola em vez de uma reta.

Ao contrário do esperado, o erro quadrático médio cresceu com o aumento da distribuição, embora o aumento tenha sido muito reduzido para ser considerado significativo. 
É provável que este tenha sido um \emph{outlier} em termos de simulação. 
Pode indicar, ainda, um viés na ferramenta de seleção pseudoaleatória utilizada para implementar a rotina \emph{Random}.


\begin{figure}[H]
	\includegraphics[width=\linewidth]{fig7_250.png}
	\centering
	
	\caption{Erro quadrático médio encontrado com $250$, $500$, $1000$ e $2000$ iterações, respectivamente, de baixo para cima.}
	\label{}
\end{figure}

%----------------------------------------------------------------------------------------


\end{document}
